% The Stratified Petersen Estimator with a Known Number of Unread Tags
% Biometrics 55, 1014-1021.
%
% Carl James Schwarz, Department of Statistics and Mathematics, Simon Fraser University
% For further information please contact me at:
%     Phone:  (604) 291-3376
%     Fax:    (604) 291-4947
%     Email:  cschwarz@cs.sfu.ca
%
% This paper contains tables that are printed in landscape orientations.
% The TeX commands \vsize and \hsize are set just before the tables are generated.
%


\input template.tex    % this is the Biometrics template 
\input rotate.tex      % this is for rotating tables

\def\E{\mathop{\hbox{\rm E}}\nolimits}


\runninghead{Stratified-Petersen with unread tags}

\galleyline{A}{B}{54}{1}

\Title{The Stratified Petersen Estimator with a Known Number of Unread Tags}



\Author{Carl J. Schwarz}

\Affiliation{Department of Statistics and Mathematics,
Simon Fraser University, Burnaby, B.C., Canada  V5A 1S6; e-mail cschwarz@cs.sfu.ca}

\Author{Myra Andrews}

\Affiliation{Department of Statistics and Mathematics,
Simon Fraser University, Burnaby, B.C., Canada  V5A 1S6}


\Authorlast{Michael R.  Link}

\Affiliation{LGL Limited, Sidney, B.C., Canada  V8L 3Y8}



\Summary{}
 
The Petersen estimator estimator of abundance.
can be biased when the assumption of homogeneous capture
probability or homogeneous recapture probability is violated.
Often this heterogeneity is related to the time or place of
capture or recapture and if these can be stratified,
the stratified-Petersen estimator reduces the bias caused by this heterogeneity.
In some experiments, not all the recovered tagged animals can be examined, and 
only a subsample has its stratum of release and recovery determined. 
We develop methods for this modified experiment and apply them
to estimate the number of salmon returning to spawn in a river in
British Columbia, Canada.


\keywords{capture-recapture; mark-recapture; Petersen estimator; 
salmon escapement; salmon spawning}




\Introduction {1.~Introduction}

The Petersen estimator of abundance 
can be severely biased if the assumption of 
equal capture or of equal recapture probabilities among all animals
is violated.
Often, heterogeneity in capture or recapture probabilities is related to the
time when or the place where the animals are captured or recaptured.
In these cases, the Stratified-Petersen method (Darroch, 1961;
Seber, 1982;  Plante, 1990;
Banneheka, Routledge, and Schwarz, 1997; Schwarz and Taylor, 1998;
Plante, Rivest and Tremblay, 1998) can be used.

In this method, in each of $s$ strata
a known number of animals ($n_i^c$) are captured
tagged with individually 
identifiable tags,
and released.
Recaptures take place in each of $t$ strata.
Tagged animals have their tags read which identifies the stratum of release and
of recapture so that the number ($m_{ij}$) of individuals marked
in stratum $i$ and recovered in strata $j$ may be determined.
A count of untagged animals recovered in each of the recovery strata
($u_j$) is also made.

Normally, all tagged animals that are recaptured  have their tags read. However, in 
some cases, the number of tagged animals recovered is too large or it is 
logistically impossible to read all of the tags. 
For example, our paper was motivated by a study where salmon are tagged
as they return to spawn and `recoveries' are made by counting the number
of tagged and untagged fish as they pass through a fish-way at a dam
upstream. The tags are readily visible so that it is easy to count the number of tagged
fish passing 
but it is difficult to extract tagged fish using a dip net to read the tags so
only a fraction of the tags can be read to determine the tag number and hence
when the fish were initially tagged.
In general, a count of the total number
of tagged animals in the recovery samples is known, but only a random
sub-sample of the recovered tagged animals have their tags read leaving
$z_j$ animals whose stratum of release is unknown.

In this paper, we extend the Darroch-Plante model for the stratified-Petersen to account
for a known number of tags recovered but not read. We apply this model to estimate
the number of salmon returning to spawn (the salmon escapement) in the Nass
River, B.C., Canada. 

\Section{2.~Notation}

We consider the case of $s$ release strata and $t$ recovery strata.
A dot in a subscript implies summation over that subscript, e.g.,
$m_{i\bullet }=\sum\limits_{j=1}^t {m_{ij}}$.


\Subsection{2.1~Parameters}

{\parindent=2cm   % start the notation
\def\litem#1{\item{\hbox to\parindent{\enspace#1\hfill}}} 
 
\litem{$N$} total number of animals in the population over all release strata.

\litem{$\psi _i$} proportion of population in release stratum $i$ ($i=1,\ldots ,s$).
$\sum\limits_{i=1}^s {\psi _i}=1$.

\litem{$N_i$} number of animals in release stratum $i$ ($i=1,\ldots ,s$). 
$N_i=N\psi_i=( {\gamma _i+\sum\limits_{j=1}^t {\mu _{ij}\lambda _j}}
)( {1+\beta _i} )$.
$\sum\limits_{i=1}^s {N_i}=N$. 
Note that the sets $\{N,\{\psi_i\}\}$ and $\{N_i\}$ are two different
but equivalent parameterizations for the number of animals in the release strata.

\litem{$c_i ,r_j$} probability of initial capture in release stratum $i$ ($i=1,\ldots ,s$),
or capture in recovery stratum $j$ ($j=1,\ldots ,t$).

\litem{$\beta _i$} odds of not capturing an animal in 
release stratum $i$ ($i=1,\ldots ,s$).
$\beta _i={{1-c_i} \over {c_i}}$ 

\litem{$\theta _{ij}$} probability that an animal released in stratum $i$
moves to recovery stratum $j$ ($i=1,\ldots ,s$; $j=1,\ldots ,t$). We allow 
$\sum\limits_{j=1}^t {\theta _{ij}}\le 1$
to account for movement to areas not sampled or mortality between release
and recovery.

\litem{$\mu _{ij}$} expected number of animals that 
are released in stratum $i$, move to recovery stratum $j$, and are recaptured
($i=1,\ldots ,s$; $j=1,\ldots ,t$). 
$\mu _{ij}=N\psi _ic_i\theta _{ij}r_j$ 

\litem{$\lambda _j$} probability that a tagged animal recovered in stratum $j$ 
will have its tag read ($j=1,\ldots ,t$). 

\litem{$\gamma _i$} expected number of animals released in 
stratum $i$ which do not have their tags read
($i=1,\ldots ,s$).
$\gamma _i=N\psi _ic_i-\sum\limits_{j=1}^t {\mu _{ij}\lambda _j}$ 

\par} % end of indent for section 2.1

\Subsection{2.2~Statistics}

{\parindent=2cm   % start the notation
\def\litem#1{\item{\hbox to\parindent{\enspace#1\hfill}}} 

\litem{$n_i^c$}  number of animals captured, tagged, and released in stratum $i$
($i=1,\ldots ,s$). 

\litem{$m_{ij}$} number of animals tagged and released in stratum $i$ that 
are recovered in stratum $j$ and have have their tags read
($i=1,\ldots ,s$; $j=1,\ldots ,t$). 

\litem{$u_j$} number of animals recovered in stratum $j$ without tags
($j=1,\ldots ,t$).

\litem{$z_j$}  number of tagged animals recovered in stratum $j$ whose
tags are not read
($j=1,\ldots ,t$). 
\par} % end of notation section

  

\Section{3.~Model Development and Fitting}

The statistics for this experiment can be arranged into a rectangular
array as shown in Table 1.
The key difference between this experiment 
and the usual stratified-Petersen experiment
is that 
the stratum of release for the 
$\{z_j\}$
tagged animals recovered is not known.

We make the same assumptions as outlined in Darroch (1961),
Seber (1982), Plante (1990),
Banneheka, Routledge, and Schwarz (1997), Schwarz and Taylor (1998), and
Plante, Rivest, and Tremblay (1998). 
In addition, we assume that the animals selected to have their tags read
are a random sample from all tagged animals recovered. 
The expected values of the statistics are shown in Table 2.

Plante (1990) and Plante, Rivest, and Tremblay (1998)
 showed that in the ordinary stratified-Petersen (where all
ags are read), not all parameters are identifiable.
We can rewrite the expected values
shown in Table 2 in terms of new identifiable parameters as shown in Table 3.
[Note that our $\gamma _i$ is defined slightly differently than
these previous works.]
In the case where $s\le t$, 
the population sizes in the release strata are identifiable
even if the population is not closed, i.e., some animals may `leave
the population' (e.g. die between release and recovery, or emigrate to
recovery strata not sampled).
This corresponds to many real situations as exemplified by
Schwarz and Taylor (1998). 
[In the case  $s\ge t$, only the population size in the recovery
strata are estimable.]

We chose to parameterize in terms of the $s$ parameters
$\{\gamma _i\}$ 
rather than the set 
$\{ N,\{\psi _i\}_{i=1,\ldots s}\}$
(recall that $\sum\limits_{i=1}^s {\psi _i}=1$) because, as will
be shown later, the estimating equations are simpler.
There is a 1--1 transformation between these two parameter sets.

Up to this point, the development has paralleled that in Plante (1990)
and Plante, Rivest, and Tremblay (1998). 
The distribution of the statistics 
$\{ m_{ij} \}$,
$\{ {n_i^c - m_{i \bullet }} \}$,
and
$\{ {u_j} \}$ 
can be modelled either as 
multinomial counts conditional upon the $n_\bullet ^c$
using the  expectations
as shown in Table 3 (as was done by Plante, 1990, and Plante, Rivest, and Tremblay,  1998)
or as independent Poisson counts - both lead to the same MLEs.

However, at this point, difficulties arise in
developing a likelihood function. 
The distribution of 
$\{ {z_j} \}$ 
conditional upon the set 
$\{ {{n_i^c - m_{i \bullet }}} \}$
is a convolution of $s$ multinomial
distributions with complex  cell probabilities and 
involves a high-dimensional summation that is not amenable
to inclusion in a likelihood, i.e this portion of the likelihood is:

%$$\eqalign{%
%L( {\left\{ {z_j} \right\}} |&{ \left\{ {n_i^c-m_{i\bullet }} \right\},
%\left\{ {\lambda _j} \right\},\left\{ {\mu_{ij}} \right\},\left\{ {\gamma _i} \right\}    } )=\cr  % {} match so far
%&\sum\limits_{z_{11}^*}{ } 
%\cdots
%\sum\limits_{z_{st}^*}{ }
%\prod\limits_{i=1}^s {\left( {\matrix{{n_i^c-m_{i\bullet }}{z_{i1}^*,\cdots ,z_{it}^*,n_i^c-m_{i\bullet }-z_{i\bullet }^*} }} \right)}
%\left[ {\prod\limits_{j=1}^t {\left( {{{\mu _{ij}\left( {1-\lambda _j} \right)} \over {\gamma _i}}} \right)^{z_{ij}^*}}} \right] 
%\times \cr
%&\left( {{
%{\gamma _i- \sum\limits_{j=1}^t {\mu _{ij}\left( {1-\lambda _j} \right)}}
% \over {\gamma _i}
%}} \right)
%^{n_i^c-m_{i\bullet }-z_{i\bullet }^*}\cr
%}$$ % end of equation alignment

$$\eqalign{%
L( {\left\{ {z_j} \right\} |& \left\{ {n_i^c-m_{i\bullet }} \right\},
\left\{ {\lambda _j} \right\},\left\{ {\mu_{ij}} \right\},\left\{ {\gamma _i} \right\}} )=\cr
&\sum\limits_{z_{11}^*} {\cdots \sum\limits_{z_{st}^*}
{\prod\limits_{i=1}^s {\left( {\matrix{{n_i^c-m_{i\bullet }}\cr
{z_{i1}^*,\cdots ,z_{it}^*,n_i^c-m_{i\bullet }-z_{i\bullet }^*}\cr
}} \right)}}}
\left[ {\prod\limits_{j=1}^t
{\left( {{{\mu _{ij}\left( {1-\lambda _j} \right)} \over {\gamma _i}}}
\right)^{z_{ij}^*}}} \right] \times \cr
&\left( {{{\gamma _i-\sum\limits_{j=1}^t
{\mu _{ij}\left( {1-\lambda _j} \right)}} \over {\gamma _i}}} \right)
^{n_i^c-m_{i\bullet }-z_{i\bullet }^*}\cr
}$$ % end of equation alignment

subject to:
$z_{\bullet j}^*=z_j,\ and\ \ z_{i\bullet }^*\le n_i^c-m_{i\bullet }.$



For these reasons, we use a generalized estimating equation (GEE) approach
(Liang and Zeger, 1986).
GEEs
have been used in capture-recapture contexts (e.g. Becker, 1984; Yip 1991)
through estimating functions based on martingales describing the 
capture and release process. In this context we will derive 
estimating equations based on the moments of the observed 
statistics.

Using the GEE approach, the estimates are derived as the solutions to:
${\bf D^{T}V}^{-1} ( {{\bf Y}-\bpi} )={\bf 0}$
where:
{\bf Y} is a vector of observed statistics;
{$\bpi$} is a 
vector of the expected values of ${\bf Y}$;
{\bf V} is a `working' covariance
matrix of the observed statistics;
and {\bf D} is matrix of partial derivatives
${{\partial \bpi } \over {\partial \bomega ' }}$ where $\bomega$ is the 
parameter set.
The estimates are consistent even if ${\bf V}$ is not the true covariance
matrix of the observed statistics and will be efficient if ${\bf V}$ is close to the 
true covariance of ${\bf Y}$.

In this case, the $(st+s+2t+1)$ components of ${\bf Y}$ are: 
$st$ components referring to 
$\{ m_{ij} \}$; 
$s$ components referring to 
$\{ {n_i^c - m_{i \bullet }} \}$;
$t$ components referring to 
$\{ u_j \}$;
$t$ components referring to 
$\{ z_j \}$;
and 1 component referring to  $n_\bullet^c - m_{\bullet \bullet }-z_\bullet $.
The components of $\bpi$ refer to the expectations of these statistics as shown
in Table 3. 
The components of $\bomega$ refer to the $st+2s+t$ parameters:
$\{ {\mu _{ij}} \},\{ {\gamma _i} \}, \{ {\beta _i} \}$, and $\{ {\lambda _j} \}$.
The working covariance matrix is a diagonal matrix
with diagonal elements corresponding to the expected values -- this 
makes the `assumption' that the observed statistics are approximately
Poisson distributed.
This formulation leads to 
estimates that minimize
the usual Pearson-$\chi^2$ measure of goodness-of-fit, i.e.
minimum $\chi^2$ estimates. 
Consequently, we expect our estimators to be nearly as efficient
as the true MLEs.

This gives rise to the following estimating equations:

{\parindent=3cm  % score equation formatting equations
\item{Parameter} Equation
 
\item{${\left\{ {\mu _{ij}} \right\}_{j=1,\ldots ,t}^{i=1,\ldots ,s}}$} 
$0=\left( {{{m_{ij}} \over {\mu _{ij}\lambda _j}}-1} \right)\lambda _j
+\left( {{{u_j} \over {\sum\limits_{a=1}^s {\beta _a\mu _{aj}}}}-1} \right)\beta _i
+\left( {{{z_j} \over {\left( {1-\lambda _j} \right)\mu _{\bullet j}}}-1} \right)\left( {1-\lambda _j} \right)$\hfill\break 
\hfill${}-\left( {{{n_\bullet ^c-m_{\bullet \bullet }-z_\bullet } \over {\gamma _\bullet
-\sum\limits_{a=1}^s {\sum\limits_{b=1}^t {\left( {1-\lambda _b} \right)\mu _{ab}}}}}-1} \right)\left( {1-\lambda _j} \right) $ 

\item{${\left\{ {\gamma _i} \right\}_{}^{i=1,\ldots ,s} }$} 
${0=\left( 
{{{n_i^c-m_{i\bullet }} \over {\gamma _i}}-1} \right)+
\left( {{{n_\bullet ^c-m_{\bullet \bullet }-z_\bullet } \over 
{\gamma _\bullet-\sum\limits_{a=1}^s {\sum\limits_{b=1}^t {\left( {1-\lambda _b} 
\right)\mu _{ab}}}}}-1} \right)} $ 

\item{${\left\{ {\beta _i} \right\}_{}^{i=1,\ldots ,s}}$}
${0=\sum\limits_{j=1}^t {\left[ {\left( {{{u_j} \over {\sum\limits_{a=1}^s {\beta _a\mu _{aj}}}}-1} \right)\mu _{ij}} \right] } } $ 


\item{${\left\{ {\lambda _j} \right\}_{}^{j=1,\ldots ,t}} $}
$0=\sum\limits_{i=1}^s {\left[ {\left( {{{m_{ij}} \over {\mu _{ij}\lambda _j}}-1} \right)\mu _{ij}} \right]}-\left( {{{z_j} \over {\left( {1-\lambda _j} \right)\mu _{\bullet j}}}-1} \right)\mu _{\bullet j}$\hfill\break
\hfill ${}+\left( {{{n_\bullet ^c-m_{\bullet \bullet }-z_\bullet } \over {\gamma _\bullet -\sum\limits_{a=1}^s {\sum\limits_{b=1}^t {\left( {1-\lambda _b} \right)\mu _{ab}}}}}-1} \right)\mu _{\bullet j} $ 

\par} % end of small table

This system of equations  does not have an analytical solution and must
be solved numerically. However, two approximate solutions are:
$\hat \gamma _i\approx n_i^c-m_{i\bullet }$ and 
$\hat \lambda _j\approx {{m_{\bullet j}} \over {m_{\bullet j}+z_{\bullet j}}}$.
The former is analogous to the results in Plante, Rivest, and Tremblay (1998), and
the latter is intuitively appealing because it represents the ratio of the number
of tags read to the total number of tags recovered in each recovery
stratum.

Under this approximate solution, the remaining equations reduce to:
$${0={{m_{ij}} \over {\mu _{ij}}}+{{u_j\beta _i} \over
{\sum\limits_{a=1}^s {\beta _a\mu _{aj}}}}-\beta _i+{{z_j} 
\over {\mu _{\bullet j}}}-1 }  $$
$${0=\sum\limits_{j=1}^t {\left[ {\left( {{{u_j} \over {\sum\limits_{a=1}^s {\beta _a\mu _{aj}}}}-1} \right)\mu _{ij}} \right]}}$$
which again are analogous to those in Plante (1990) with her $m_{ij}$ 
`replaced' by 
$m_{ij}+z_j{{\mu _{ij}} \over {\mu _{\bullet j}}}$ representing an `estimate'
of the number of tags released in stratum $i$ and recovered in
stratum $j$ had all tags been read. 
The last equation has the same form as Plante (1990).
As outlined in Schwarz and Taylor (1998), the
$\{ {\beta _i} \}$
are found to minimize the discrepancy between
the $u_j$ row and the linear combination of the rows of the  $\mu_{ij}$ matrix.

Initial values for the $\beta_i$ can be found using a least-squares approach
(Banneheka, Routledge, and Schwarz, 1997)
after adjusting the $m_{ij}$ to account for unread tags.

The variances of the estimated parameters are found as:
$V\left( {\hat \bomega} \right)={\bf \left[ {D^{T}V^{-1}D} \right]}^{-1} %***left/right left in
\left[ { {\bf D^{T}V}^{-1}V({\bf Y}){\bf V}^{-1}{\bf D}} \right]         %***left/right left in
\left[ { {\bf D^{T}V}^{-1}{\bf D}} \right]^{-1}$                         %***left/right left in
where $V({\bf Y})$ is the true variance-covariance matrix of {\bf Y}.
The $V({\bf Y})$ can be readily derived
despite the difficulties in writing a likelihood for the $z_j$.
The variance of the estimates can be estimated by replacing the parameters 
by their estimates in the
above equation.



Estimates of the initial stratum sizes are then found as:
$\hat N_i=( {\hat \gamma _i+\sum\limits_{j=1}^t {\hat \mu _{ij}\hat \lambda _j}} )( {1+\hat \beta _i} )$.
The estimated total population size is found by summing the individual stratum
estimates:
$\hat N=\sum\limits_{i=1}^s {( {\hat \gamma _i+\sum\limits_{j=1}^t {\hat \mu _{ij}\hat 
\lambda _j}} )( {1+\hat \beta _i} )}$.

The estimated
variance of these derived parameters is found using the delta-method 
and the estimated variances of the original parameters.

Note that
$\hat \gamma_i \approx n_i^c-m_{i\bullet }$,
$\widehat{\mu _{ij}\lambda _j}\approx m_{ij}$, and,
$1+\hat \beta _i={1 \over \hat {c}_i}$ so that
$\hat N \approx \sum\limits_{i=1}^s {{{n_i^c} \over {\widehat{c}_i}}}$,
a Horvitz-Thompson type estimator.
Huggins (1989) and Yip, Huggins, and Lin (1996) examined the
performance of similar-types of estimators for closed
populations and found that these performed
poorly when sample sizes were small and the capture
probabilities poorly estimated. 
This is also expected to be true for our estimators as confirmed
by our simulation study.



A goodness-of-fit statistic can be found using a Pearson-type statistic as:
$$\displaylines{   % break formula into smaller bits
{\rm X}^2=\sum\limits_{i=1}^s {\sum\limits_{j=1}^t 
{{{({m_{ij}-\hat \mu _{ij}\hat \lambda _j})^2} \over {\hat \mu _{ij}\hat \lambda _j}}}}
+\sum\limits_{i=1}^s {{{({n_i^c-m_{i\bullet }-\hat \gamma _i})^2} \over {\hat \gamma _i}}}
+\sum\limits_{j=1}^t {{{\left( {u_j-\sum\limits_{a=1}^s {\hat \beta _a\hat \mu _{aj}}} \right)^2} \over {\sum\limits_{a=1}^s {\hat \beta _a\hat \mu _{aj}}}}}\hfill\cr  % break and justify left
\hfill {}
+\sum\limits_{j=1}^t {{{\left( {z_j-\hat \mu _{\bullet j}({1-\hat \lambda _j})} \right)^2} \over {\hat \mu _{\bullet j}({1-\hat \lambda _j})}}}
+{{\left( {n_\bullet ^c-m_{\bullet \bullet }-z_\bullet -\left( {\hat \gamma _\bullet -\sum\limits_{a=1}^s {\sum\limits_{b=1}^t {\hat \mu _{ab}({1-\hat \lambda _b})}}} \right)} \right)^2} \over {{\hat \gamma _\bullet-\sum\limits_{a=1}^s {\sum\limits_{b=1}^t {\hat \mu _{ab}({1-\hat \lambda _b}) }}}  }}\cr
}$$                  % end of displaylines

This will have an approximate $\chi _{t-s}^2$ distribution.
Not all of the cells
are independent of each other because there is some double counting of
animals between the $z_j$ and the $n_i - m_{i \bullet}$. 
However, the $z_j$ will be conditionally independent
of the other cells and so in large samples the approximation should be valid.

Models where constraints are imposed upon the parameters (e.g. equal tag reading
rate in all recovery strata) can be fit by using variants of the chain
rule.

Because there is no formal likelihood, there is no simple procedure for model 
selection and testing, except, perhaps, comparing the change in the
goodness of fit statistic. 
Williams (1970) outlined a bootstrap-type procedure for this that
could be used to discriminate between models.
Wald-type tests could also be constructed
using the estimates and the estimated variance-covariance matrix.

As noted by Schwarz and Taylor (1998), extensive pooling of the rows
or column counts may be required, and there is no objective method
of assessing which poolings are `optimal'.
Schwarz and Taylor (1998) also noted that the estimates of $\beta_i$
are very unstable 
if the expected counts are small 
leading to unstable estimates for $N$;
they suggest that the expected
values of $m_{ij}$ should not be less than 5-10. 

Specialized software to fit the above model using the estimating equations
has been written in S-Plus and
is available from the first author. This software also will fit simpler models 
where the tag-reading rate is assumed to be equal for all the recovery 
strata, where the tag application rate is equal for all release
strata,
or where the parameters can be modelled using covariates.
In the middle model above, the estimate of the total population size
is algebraically equal to the simple Petersen estimator which is known to be
consistent when all the initial capture probabilities are equal.

In the absence of specialized software for this experiment, approximate
solutions can be found using software for the stratified-Petersen with all
tags read (e.g. SPAS from Arnason et al., 1996) by initially
`distributing' the
unread tags into their respective rows in the same proportion as the $m_{ij}$ 
to their
column sums. 
Then use SPAS to obtain estimates of the $\mu _{ij}$; use these
revised estimates to reapportion the $z_j$; and repeat until convergence.
Of course, the final variance estimates reported by SPAS will be too small.





\Section{4.~Example}

The Nass River is located in northern British Columbia, Canada and
supports several stocks of sockeye salmon. 
From mid-June to early September 1995, 
daily samples of up to 400
returning sockeye salmon 
were captured with two fish wheels, 
tagged with individually numbered tags, and released
back to the river (Link and Gurak, 1997).
The fish migrate up river to spawn at several sites.
The fish arrive at the entrance to one spawning area, Meziadin Lake,
approximately 3 weeks later.
Here fish travel through a fishway 
and counting chute 
where they can be observed, counted,
and fish removed using a dipnet to read their tags. 
Approximately 1,500-10,000 fish  pass through the fishway each day.
Because the tags are quite visible, it is relatively easy
to count the number of tagged fish, but it is more 
difficult to recapture the fish, and not all tagged
fish can be captured to be read.

Here, the population is not `closed' as fish can be removed by
a fishery after passing the fish wheels and spawn in 
many other sites not sampled. However, we can
still obtain estimates of the number of fish passing
the fish wheels.
Because the $m_{ij}$ matrix based on the daily counts is quite
sparse, releases and recoveries were pooled into 8 release and
10 recovery strata as shown in Table 4.

The estimates of the parameters from the full model with separate
initial capture probabilities for each release stratum and 
separate tag reading rates for each recovery stratum are shown
in Table 5.
Although the goodness-of-fit statistic shows some lack of fit
(${\rm X}^2$ = 16.4 with 2 df),
the
majority of the lack-of-fit comes from 2 cells, the
deviations between the observed and expected counts are usually below 4\%, and
large samples sizes make even minor deviations detectable.
The estimates of the $\lambda _j$ indicate that a simpler
model would not be tenable (the goodness-of-fit statistic increases
to 1008.2 with 11 df). The estimates of
the $\beta _i$ indicate that tagging rates varied from about 
1/90 per day to 1/30 per day and again a simpler
model is not tenable (the goodness-of-fit statistic increases
to 433.6 with 9 df). 

The estimate of the overall run is 356,104 fish with an 
estimated se of 4,749 fish. Note that the
estimated se of the overall run is comparable to the se
of the estimated individual strata sizes - this is caused
by the fact that the latter are very highly correlated.


The pooled-Petersen (i.e. ignoring
stratification) is approximately 3.7\% smaller. 
The apparent bias is statistically significant, but not of
biological significance.
The negative bias is not unexpected. As noted by Seber (1982, Section 3.2.2, p. 86) and
by Schwarz and Taylor (1998), a positive correlation
between the capture and recapture probabilities among
fish leads to a negative bias in the Petersen estimate.
Ironically, the capture-probability of the fishwheel-caught fish is often
the highest when large numbers of fish are migrating due to 
optimal water conditions and what might
be density dependent catchability may be positively
correlated with abundance.
Similarly, at the fishway, more effort is expended when
a large number of fish are expected to arrive and the
capture probability is also larger!


Two simulation studies were performed to (1) asses the performance
of the proposed estimator in the current example, and (2)
to assess the performance of the estimators when recapture
rates are much smaller and data is sparser.

One hundred simulated datasets from a population based upon the
estimates in Table 5(a) were generated and
and estimates and estimated standard 
errors were obtained for each data set.
All point estimates appear to unbiased but
there may be a slight underestimate of the actual standard
error for the strata sizes and the overall population size.
The good performance of the estimators is not unexpected given the
relatively large number of tagged fish recovered.

A second set of simulated datasets with approximately $1/10$ of the recovery
effort observed in the example was generated using a similar procedure.
The point estimators remained unbiased, but as expected, the variability 
of the estimators increased by about a factor of $\sqrt{10}$. 
The estimated standard errors are
much more variable and there appears to be significant bias in the
estimated standard errors for the stratum and overall population sizes.
In sparse data, many of the $m_{ij}$ are
zero, and Plante, Rivest, and Tremblay (1998) note that the
corresponding $\hat{\mu}_{ij}=0$, and these sampling zeroes
have the same effect as structural zeroes in the variance estimation, i.e.
the observed `zeroes' are treated as
known parameters and so the estimated variances are
biased downwards. 


\Section{5.~Discussion}

In this experiment, the use of estimating equations provides a
simple method to obtain estimates compared to using
a formal likelihood which is not tractable.
As noted in the paper, our estimates are a form of
minimum $\chi^2$ estimators, and asymptotically
will be close to fully efficient.
The disadvantage of this approach is that model
selection (through likelihood ratio tests and AIC) is
not readily done, although some modification
using changes in the goodness-of-fit statistic as a 
surrogate for changes in the likelihood could be
possible.
Our approach is also related to that of 
quasi-likelihood
(Wedderburn, 1974). 

The estimates could also have been obtained
using the EM-algorithm where the missing data
refers to the allocation of tags counted but not
read to the cells in the $m_{ij}$ array.
The evaluation and maximization of the complete-data
likelihood then then exactly the same as that
of Plante, Rivest, and Tremblay (1998). However,
it is still difficult to evaluate the expectation 
of the complete-data likelihood over the unobservable
components subject to the complex observed 
constraints. However, an approximate solution
can be obtained as outlined at the end of Section 3.1.

It is interesting to estimate the effect of not reading all
the tags upon the estimates. 
If we replace the observed $m_{ij}$ by 
$m_{ij}^{*} = m_{ij}+z_j{{\hat \mu _{ij}} \over {\hat \mu _{\bullet j}}}$ 
to simulate what would happen if all tags were read, then
analyze this simulated date using SPAS, the final estimates
of the population size are unchanged, but the se of the
overall run size decreases by approximately 8\%.
It is not likely cost effective to increase the sampling
effort to get this increased precision in this experiment. 
Using our software, simulated data could be analyzed
to help select the optimal level of sampling at
both the capture and recapture stages.

The ability to partially count tagged recoveries has
two other immediate applications.
First, it is quite common in fisheries to tag fish
by injecting them with a small wire about 2 mm in 
size that is coded with information about the time
and location of release and to simultaneously
batch mark them with a fin clip.
When fish are recovered, it is relatively easy to 
identify the marked fish from the fin clip, but more tedious and
time consuming to extract and read the coded-wire
tag. For example, recoveries are often obtained 
from recreational fishers who deposit the head into
specially marked barrels at fish cleaning stations and
these heads are dissected at the laboratory.
Now only a portion of the fish need to be examined
further but the total count of the recoveries
can be incorporated.

Second, field conditions may make it difficult
to reliably identify tagged fish. Using methods similar
to those outlined in Rajwani and Schwarz (1997),
a second survey could be conducted to estimate the number
of tags overlooked - i.e. an estimate of the $z_j$
could be obtained.
These could be used to improve the estimates
based on the initial, faulty, survey - but of 
course, the variance of the estimates needs to be
adjusted in a similar fashion as in Rajwani and
Schwarz (1997).

Lastly, in this particular experiment, the
pooled-Petersen estimator had a statistically
significant but biologically meaningless negative
bias. However, it is not difficult to 
construct examples where the bias can be
as great as 40\%.
As well, the stratified-Petersen provides estimates
of the capture-efficiency of the fish wheels
which 
are being used to evaluate the performance of an in season
mark-recapture method (Link and Gurak, 1997).
Similarly, Schwarz and Taylor (1998) compared
estimates from a stratified mark-recapture experiment
to those obtained by a hydro-acoustic methods to try
and ascertain why the estimates of the total run
varied by a factor of 2 between the two methods.



\Acknowledgments{} 

This work was supported by a Natural Science and Engineering
Research Council of Canada (NSERC) Research Grant to CJS.
The sockeye salmon data were provided by the Nisga'a
Tribal Council and LGL Limited.


\resume{}

To be added by editor.



\References{}

\ref Arnason, A. N., Kirby, C. W., Schwarz, C. J., and Irvine, J. R. (1996). 
Computer analysis of marking data from stratified populations for estimation of 
salmonid escapements and the size of other populations. 
{\it Canadian Technical Report of Fisheries and Aquatic Sciences} {\bf 2106}.

\ref Banneheka, S. G., Routledge, R. D., and Schwarz, C. J.
(1997).  
Stratified two-sample tag-recovery census of closed populations.
{\it Biometrics\/}
{\bf 53}, 1212--1224.

\ref Becker, N. G. (1984).
Estimating population size from capture-recapture experiments in continuous
time.
{\it Australian Journal of Statistics\/} {\bf 26}, 1-7.

\ref Darroch, J. N. (1961).  
The two-sample capture-recapture census when tagging and sampling are stratified.
{\it Biometrika\/} {\bf 48}, 241--260.

\ref Huggins, R. M. (1989).
On the statistical analysis of capture-experiments.
{\it Biometrika\/} {\bf 76}, 133-140.

\ref Liang, K. Y. and Zeger, S. L. (1986).
Longitudinal data analysis using generalized linear models.
{\it Biometrika\/} {\bf 73}, 13-22.

\ref Link, M. R. and Gurak, A. C. (1997).
The 1995 Fishwheel Project on the Nass River, B.C.
{\it Canadian Manuscript Report of Fisheries and Aquatic Sciences\/}
{\bf 2422}, xi+99 p.

\ref Plante, N. (1990). 
Estimation de la taille d'une population animale \`a l'aide d'un
mod\`ele de capture-recapture avec stratification. M.Sc. Thesis, Universit\'e
 Laval, Canada.

\ref Plante, N., Rivest, L.-P., and Tremblay, G. (1998). 
Stratified capture-recapture estimation of the size of a closed population.
{\it Biometrics\/} {\bf 54}, 47--60.

\ref Rajwani, K. and Schwarz, C. J. (1997). 
Adjusting for missing tags in salmon escapement surveys. 
{\it Canadian Journal of Fisheries and Aquatic Sciences\/} {\bf 54}, 800--808.

\ref Schwarz, C. J. and Taylor, C. G. (1998). 
The use of the stratified-Petersen estimator in fisheries management: estimating the number of pink salmon (Oncorhynchus gorbuscha) spawners in the Fraser River.
{\it Canadian Journal of Fisheries and Aquatic Sciences\/} {\bf 55},
281--297.

\ref Seber, G. A. F. (1982).
{\it The Estimation of Animal Abundance and Related Parameters}, 2nd edition.
London:Griffen.

\ref Wedderburn, R. W. M. (1974).
Quasi-likelihood functions, generalized linear models, and the Gauss-Newton
method. {\it Biometrika\/} {\bf 61}, 439-437.

\ref Williams, D. A. (1970).
Discussion on ``A method of discriminating between models'',
{\it Journal of the Royal Statistical Society, Series B\/} {\bf 32}, 350.

\ref Yip, P. (1991).
A martingale estimating equation for a capture-recapture experiment in discrete
time.
{\it Biometrics\/} {\bf 47}, 1081-1088.

\ref Yip, P. S., Huggins, R. M., and Lin, D. Y. (1996).
Inference for capture-recapture experiments in continuous time with
variable capture rates.
{\it Biometrika\/} {\bf 83}, 477-483.



\vfill\eject  % new page
%-------------------------------------------------------------------------------------
% This table will be printed in landscape orientation
 
\newbox\obsstat
\setbox\obsstat=\vbox\bgroup
\hsize=54pc



\centerline{\bf Table 1}

\tabcaption{Observed statistics}

{  % start of table
\tabskip=0pt\baselineskip=10pt
\def\tablerule{\noalign{\vskip3pt\hrule\vskip3pt}}
\def\tabrule{\noalign{\hrule\vskip1.5pt\hrule\vskip3pt}}

\halign{  % first define the types of columns in the table
   \hfil#\hfil&            % center justified - stratum number
   \hfil#\hfil \tabskip=40pt &   % right justified  - number released
   \hfil#\hfil&                 % number recovered in recovery stratum 1
   \hfil#\hfil&         %  ... 
   \hfil#\hfil&         %  ...
   \hfil#\hfil&         %  t
   \hfil#\hfil%         % total recoveries - note the terminating %
   \cr             % end of preamble
\tabrule  % double line after table caption
Release & Number &    \multispan4 \hfill Recovery Stratum\hfill    & Tags not \cr
Stratum & released & $1$   & $2$   &  $\ldots$  &  $t$  &  read     \cr
\noalign{\vskip-2pt}
\tablerule
1      &  $n_1^c$ & $m_{11}$ & $m_{12}$ & $\ldots$ & $m_{1t}$ & $n_1^c - m_{1 \bullet}$ \cr
2      &  $n_2^c$ & $m_{21}$ & $m_{22}$ & $\ldots$ & $m_{2t}$ & $n_2^c - m_{2 \bullet}$ \cr
$\ldots$ & $\ldots$   & $\ldots$   & $\ldots$   & $\ldots$ & $\ldots$   & $\ldots$      \cr
$s$    &  $n_s^c$ & $m_{s1}$ & $m_{s2}$ & $\ldots$ & $m_{st}$ & $n_s^c - m_{s \bullet}$ \cr
\noalign{\vskip 6pt}
\multispan2 Number recovered \hfill \cr
\multispan2 without tags \hfill & $u_1$ & $u_2$ & $\ldots$ & $u_t$ \cr
\noalign{\vskip 6pt}
\multispan2 Number tags recovered \hfill  \cr
\multispan2 but not read \hfill& $z_1$ & $z_2$ & $\ldots$ & $z_t$  \cr
\noalign{\vskip-2pt}
\tablerule
\multispan5 Number of tags not recovered =
$n_{\bullet}^c - m_{\bullet \bullet} - z_{\bullet}$ \hfill \cr
}   % end of halign
}   % end of table
\egroup   % end of hbox definition for Table 1

\rotl{\obsstat}

\vfill\eject  % new page
%-------------------------------------------------------------------------------------


% This table will be printed in landscape orientation
 
\newbox\eval
\setbox\eval=\vbox\bgroup
\hsize=54pc

\centerline{\bf Table 2}

\tabcaption{Expected values of statistics shown in Table 1}

{  % start of table
\tabskip=0pt\baselineskip=10pt
\def\tablerule{\noalign{\vskip3pt\hrule\vskip3pt}}
\def\tabrule{\noalign{\hrule\vskip1.5pt\hrule\vskip3pt}}

\halign{  % first define the types of columns in the table
   \hfil#\hfil&    % centre justified - stratum number
   \hfil#\hfil     &   % right justified  - number released
   \hfil#\hfil&         % number recovered in recovery stratum 1
   \hfil#\hfil&         %  ...
   \hfil#\hfil&         %  t
   \hfil#\hfil%         % total recoveries - note the terminating %
   \cr             % end of preamble
\tabrule  % double line after table caption
Release & Number &
\multispan3 \hfill Recovery Stratum\hfill    
& Tags not \cr
Stratum & released & $1$   
&  $\ldots$  &  $t$  &  read     \cr
\noalign{\vskip-2pt}
\tablerule
1      &  $N\psi _1c_1$
& $N\psi _1c_1\theta _{11}r_1\lambda _1$ 
& $\ldots$ 
& $N\psi _1c_1\theta _{1t}r_t\lambda _t$ 
& $N\psi _1c_1-\sum\limits_{j=1}^t {N\psi _1c_1\theta _{1j}r_j\lambda _j}$
 \cr
2      &  $N\psi _2c_2$ 
& $N\psi _2c_2\theta _{21}r_1\lambda _1$ 
& $\ldots$ 
& $N\psi _2c_2\theta _{2t}r_t\lambda _t$ 
& $N\psi _2c_2-\sum\limits_{j=1}^t {N\psi _2c_2\theta _{2j}r_j\lambda _j}$
\cr
$\ldots$ & $\ldots$   
& $\ldots$   & $\ldots$ & $\ldots$   &$ \ldots$    \cr
$s$    &  $N\psi _sc_s$ 
& $N\psi _sc_s\theta _{s1}r_1\lambda _1$ 
& $\ldots$ 
& $N\psi _sc_s\theta _{st}r_t\lambda _t$ 
& $N\psi _sc_s-\sum\limits_{j=1}^t {N\psi _sc_s\theta _{sj}r_j\lambda _j}$
 \cr
\multispan2 Number recovered \hfill \cr
\noalign{\vskip-6pt}
\multispan2 without tags \hfill  & 
$\sum\limits_{i=1}^s {N\psi _i ({1-c_i})\theta _{i1}r_1}$ 
&$\ldots$ 
&$\sum\limits_{i=1}^s {N\psi _i({1-c_i})\theta _{it}r_t}$  \cr
\noalign{\vskip 6pt}
\multispan2 Number tags recovered \hfill 
&$\sum\limits_{i=1}^s {N\psi _i c_i \theta _{i1}r_1} ({1-\lambda _1})$ 
&$\ldots$ 
&$\sum\limits_{i=1}^s {N\psi _i c_i \theta _{it}r_t} ({1-\lambda _t})$  \cr
\noalign{\vskip-9pt}
\multispan2 but not read \hfill \cr
\noalign{\vskip-2pt}\tablerule
\multispan5 Expected number of tags not recovered = 
$\sum\limits_{i=1}^s { {N\psi _ic_i} } [{1-\sum\limits_{j=1}^t {\theta _{ij}r_j}}]$ \hfill \cr
}   % end of halign
}   % end of table

\egroup   % end of hbox definition for Table 2

\rotl{\eval}


\vfill\eject  % new page
%-------------------------------------------------------------------------------------


% This table will be printed in landscape orientation
 
\newbox\evalnew
\setbox\evalnew=\vbox\bgroup
\hsize=54pc
\centerline{\bf Table 3}

\tabcaption{Expected value of observed statistics in terms of the identifiable
parameters}

{  % start of table
\tabskip=0pt\baselineskip=10pt
\def\tablerule{\noalign{\vskip3pt\hrule\vskip3pt}}
\def\tabrule{\noalign{\hrule\vskip1.5pt\hrule\vskip3pt}}

\halign{  % first define the types of columns in the table
   \hfil#\hfil&    % center justified - stratum number
   \hfil#\hfil \tabskip=40pt &   % right justified  - number released
   \hfil#\hfil&         % number recovered in recovery stratum 1
   \hfil#\hfil&         %  ...
   \hfil#\hfil&         %  t
   \hfil#\hfil%         % total recoveries - note the terminating %
   \cr             % end of preamble
\tabrule  % double line after table caption
Release & Number &    
\multispan3 \hfill Recovery Stratum\hfill    
& Tags not \cr
Stratum & released & $1$   
&  $\ldots$  &  $t$  &  read     \cr
\noalign{\vskip-2pt}
\tablerule
1      &  
$\gamma _1+\sum\limits_{j=1}^t {\mu _{1j} \lambda_j}$
&$\mu_{11}\lambda _1$
&$\ldots$ 
&$\mu_{1t}\lambda _t$ & $\gamma _1$ \cr
2      &  $\gamma _2+\sum\limits_{j=1}^t {\mu _{2j} \lambda_j}$ 
&$\mu_{21}\lambda _1 $ 
&$\ldots$ 
&$\mu_{2t}\lambda _t$ & $\gamma _2$ \cr
$\ldots$ & $\ldots$   
& $\ldots$   & $\ldots$ & $\ldots$   & $\ldots$          \cr
$s$    &  $\gamma _s+\sum\limits_{j=1}^t {\mu _{sj} \lambda_j}$ 
&$\mu_{s1}\lambda _1$ 
&$\ldots$ 
&$\mu_{st}\lambda _t$ & $\gamma _s$ \cr
\noalign{\vskip 6pt}
\multispan2 Number recovered \hfill \cr
\noalign{\vskip-6pt}
\multispan2 without tags \hfill  
&$\sum\limits_{i=1}^s {\beta _i\mu _{i1}}$ 
&$\ldots$ 
&$\sum\limits_{i=1}^s {\beta _i\mu _{it}}$ \cr
\noalign{\vskip 6pt}
\multispan2 Number tags recovered \hfill  \cr
%\noalign{\vskip-1pt}
\multispan2 but not read \hfill 
&$({1-\lambda _1})\mu _{\bullet 1}$ 
&$\ldots$ 
&$({1-\lambda _t})\mu _{\bullet t}$  \cr
\noalign{\vskip-2pt}
\tablerule
\multispan5 Expected number of tags not recovered =
$\gamma _\bullet -\sum\limits_{j=1}^t {({1-\lambda _j})\mu _{\bullet j}}$ \hfill \cr
}   % end of halign
}   % end of table

\egroup   % end of hbox definition for Table 3

\rotl{\evalnew}


\vfill\eject  % new page
%-------------------------------------------------------------------------------------

% This table will be printed in landscape orientation
 
\newbox\exdata
\setbox\exdata=\vbox\bgroup
\hsize=54pc
\centerline{\bf Table 4}

\tabcaption{Summary statistics for sockeye salmon returning to spawn in the Nass River, B.C., Canada}

{  % start of table
\tabskip=0pt\baselineskip=10pt
\def\tablerule{\noalign{\vskip3pt\hrule\vskip3pt}}
\def\tabrule{\noalign{\hrule\vskip1.5pt\hrule\vskip3pt}}

\halign{  % first define the types of columns in the table
   \hfil#\hfil&    % center justified - stratum number
   \hfil#\quad&    % right justified  - number released
   \hfil#\quad&         % number recovered in recovery stratum 1
   \hfil#\quad&         %  2
   \hfil#\quad&         %  3
   \hfil#\quad&         %  4 
   \hfil#\quad&         %  5 
   \hfil#\quad&         %  6 
   \hfil#\quad&         %  7
   \hfil#\quad&         %  8 
   \hfil#\quad&         %  9
   \hfil#\quad&         %  10
   \hfil#\quad%    % total recoveries - note the terminating %
   \cr             % end of preamble
\tabrule  % double line after table caption
Release & Number &    \multispan{10} \hfill Recovery Stratum\hfill    & Tags not \cr
Stratum & released & $1$   & $2$ & $3$ & $4$ & $5$ & $6$ & $7$ & $8$ & $9$ &  $10$  &  read     \cr
\noalign{\vskip-2pt}
\tablerule
     1 &1,070 & 118 &  68 &   8 &   7 &   1 &   0 &  0 &   0 &   0 &  0  &  868 \cr
     2 &1,919 &  32 & 245 &  88 &  45 &  14 &   3 &  2 &   2 &   0 &  0  & 1,488 \cr
     3 &2,487 &   0 & 114 & 251 & 216 &  89 &  11 &  3 &   4 &   1 &  1  & 1,797 \cr
     4 &1,103 &   0 &   0 &   6 &  93 & 221 &  65 & 19 &   9 &   1 &  0  &  689 \cr
     5 & 763 &   0 &   0 &   0 &   0 &  52 &  98 & 47 &  20 &   4 &  0  &  542 \cr
     6 & 638 &   0 &   0 &   0 &   0 &   0 &   1 & 30 & 119 &  56 &  8  &  424 \cr
     7 & 628 &   0 &   0 &   0 &   0 &   0 &   0 &  0 &  61 & 171 & 18  &  378 \cr
     8 & 209 &   0 &   0 &   0 &   0 &   0 &   0 &  0 &   0 &   8 & 38  &  163 \cr
\noalign{\vskip 6pt}
\multispan2 Number recovered \hfill \cr
\multispan2 without tags \hfill  & 
             13,047 &54,291& 33,389 &11,740 &13,411 &10,037& 8,448& 16,156& 21,178 &5,772 \cr
\noalign{\vskip 6pt}
\multispan2 Number tags recovered \hfill  \cr
\multispan2 but not read \hfill & 
           198  & 914 &  877  &  85 &   87 &  105  & 13 &   59 &  108 &  32 \cr
\noalign{\vskip-2pt}
\tablerule
\multispan4 Number of tags not recovered = 3,871.\hfill  \cr
}   % end of halign
}   % end of table


\egroup   % end of hbox definition for Table 1

\rotl{\exdata}

\vfill\eject  % new page
%-------------------------------------------------------------------------------------
% This table will be printed in landscape orientation
 
\newbox\exest
\setbox\exest=\vbox\bgroup
\hsize=54pc
\centerline{\bf Table 5 (a)}

\tabcaption{Estimates of parameters from fitting the full model}

{  % start of table
%\tabskip=0pt
\baselineskip=10pt
\def\tablerule{\noalign{\vskip3pt\hrule\vskip3pt}}
\def\tabrule{\noalign{\hrule\vskip1.5pt\hrule\vskip3pt}}

\halign{  % first define the types of columns in the table
   \hfil#\hfil&    % center justified - stratum number
   \hfil#\quad&    % right justified  - number released
   \hfil#\quad&         % number recovered in recovery stratum 1
   \hfil#\quad&         %  2
   \hfil#\quad&         %  3
   \hfil#\quad&         %  4 
   \hfil#\quad&         %  5 
   \hfil#\quad&         %  6 
   \hfil#\quad&         %  7
   \hfil#\quad&         %  8 
   \hfil#\quad&         %  9
   \hfil#\quad&         %  10
   \hfil#\quad&    % gamma i hat
   \hfil#\quad%    % beta hat % note the terminating %
   \cr             % end of preamble
\tabrule  % double line after table caption
Release &   &    \multispan{10} \hfill Estimates of $\hat \mu _{ij}$
 \hfill    & \cr
Stratum & $\hat N_i$
 & $1$   & $2$ & $3$ & $4$ & $5$ & $6$ & $7$ & $8$ & $9$ &  $10$  &  $\hat \gamma _i$ & $\hat \beta _i$   \cr
\noalign{\vskip-2pt}
\tablerule
1&36,883.4&273.6&213.8&27.5&8.9&1.2&0.0&0.0&0.0&0.0&0.0&867.4&33.5  \cr
2&102,511.6&74.1&772.0&297.7&58.9&17.1&4.4&2.9&2.3&0.0&0.0&1,487.0&52.4  \cr
3&48,792.9&0.0&357.8&875.7&271.2&109.3&17.3&3.6&5.0&1.4&1.5&1,795.8&18.6  \cr
4&31,941.5&0.0&0.0&20.8&118.1&271.0&100.2&23.8&11.1&1.5&0.0&688.5&27.9  \cr
5&35,316.7&0.0&0.0&0.0&0.0&63.6&145.6&64.4&23.6&5.9&0.0&541.6&45.2  \cr
6&56,086.6&0.0&0.0&0.0&0.0&0.0&1.4&53.3&127.6&86.1&11.8&423.7&86.7  \cr
7&31,760.4&0.0&0.0&0.0&0.0&0.0&0.0&0.0&71.2&254.5&26.7&377.7&49.6  \cr
8&12,811.1&0.0&0.0&0.0&0.0&0.0&0.0&0.0&0.0&12.0&56.3&162.9&60.3  \cr
&&  \cr
$\hat \lambda _j$
&&.431&.318&.287&.809&.812&.629&.886&.785&.690&.670&&  \cr
&&&&&&&&&&&&&   \cr
$\hat N=$ &356,104.1&&&&&&&&&&&&  \cr
&&&&&&&&&&&&&  \cr
\noalign{\vskip-2pt}
\tablerule
}   % end of halign
}   % end of table

\vskip 1cm
\centerline{\bf Table 5 (b)}

\tabcaption{Estimated se of parameter estimates from fitting the full model}

{  % start of table
%\tabskip=0pt
\baselineskip=10pt
\def\tablerule{\noalign{\vskip3pt\hrule\vskip3pt}}
\def\tabrule{\noalign{\hrule\vskip1.5pt\hrule\vskip3pt}}

\halign{  % first define the types of columns in the table
   \hfil#\hfil&    % center justified - stratum number
   \hfil#\quad&    % right justified  - number released
   \hfil#\quad&         % number recovered in recovery stratum 1
   \hfil#\quad&         %  2
   \hfil#\quad&         %  3
   \hfil#\quad&         %  4 
   \hfil#\quad&         %  5 
   \hfil#\quad&         %  6 
   \hfil#\quad&         %  7
   \hfil#\quad&         %  8 
   \hfil#\quad&         %  9
   \hfil#\quad&         %  10
   \hfil#\quad&    % gamma i hat
   \hfil#\quad%    % beta hat % note the terminating %
   \cr             % end of preamble
\tabrule  % double line after table caption
Release &   &    \multispan{10} \hfill  se ($\hat \mu _{ij}$)
 \hfill    & \cr
Stratum & $se(\hat N_i)$
 & $1$   & $2$ & $3$ & $4$ & $5$ & $6$ & $7$ & $8$ & $9$ &  $10$  &  se($\hat \gamma _i)$ & $se(\hat \beta _i)$   \cr
\noalign{\vskip-2pt}\tablerule
1&3,095.2&18.7&24.5&9.7&3.3&1.2&.0&.0&.0&.0&.0&29.4&3.1   \cr
2&5,676.8&12.3&38.3&27.0&7.2&4.5&2.6&1.8&1.7&.0&.0&38.5&3.2 \cr
3&4,735.8&.0&30.3&38.6&16.3&11.3&5.2&2.0&2.5&1.4&1.5&42.2&1.9\cr
4&3,346.3&.0&.0&8.5&11.0&16.6&11.7&5.1&3.7&1.5&.0&26.2&3.1\cr
5&3,919.9&.0&.0&.0&.0&8.5&13.2&7.8&5.4&2.9&.0&23.3&5.4\cr
6&5,535.8&.0&.0&.0&.0&.0&1.5&5.9&10.4&10.7&4.1&20.6&9.3\cr
7&3,694.9&.0&.0&.0&.0&.0&.0&.0&9.1&16.9&6.0&19.4&6.2\cr
8&2,188.8&.0&.0&.0&.0&.0&.0&.0&.0&4.1&8.2&12.8&11.3\cr
&& \cr
$se(\hat \lambda _j)$&&.027&.013&.013&.018&.018&.030&.026&.027&.024&.048&& \cr
&& \cr
$se(\hat N)=$&4,749.4& \cr
\noalign{\vskip-2pt}
\tablerule
}   % end of halign
}   % end of table

\egroup   % end of hbox definition for Table 1

\rotl{\exest}




\end  % of document
